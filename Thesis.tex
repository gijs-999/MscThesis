\documentclass[11pt]{report}

% PACKAGES 
	\usepackage[a4paper,left=28mm,right=28mm,top=30mm,bottom=30mm]{geometry}
	\usepackage{graphicx}		% Used to import external graphics (figures)
	\usepackage[table]{xcolor}	% To include colors 
	\usepackage{amsmath}   		% For most of the math symbols and environments (such as \begin{align})
	\usepackage{amssymb}		% For using symbols in the document
	\usepackage{float}			% Arranging of figures on the page
	\usepackage[bf]{caption}	% Arranging the captions in floating environments [bf] makes the Figures bold
	\usepackage{subcaption}     % To arrange captions of subfigures
	\usepackage{booktabs} 		% For standard tabular tables, with rules
	\usepackage{tabularx}	    % For clean tables such as in the Nomenclature
	\usepackage{fancyhdr} 		% Fancy headers
	\usepackage[colorinlistoftodos]{todonotes}		% To create todo notes
	\usepackage[nottoc,notlot,notlof]{tocbibind}    % Add bibliography to content
	\usepackage{bm}				% Make bold symbols
	\usepackage[auto-lang=false]{lipsum}
	\usepackage{hyperref}		% Used for referring to links inside and outside the document
% LAY-OUT 

	%Bibliography for references, with reference style options
	\usepackage[
	backend=biber,
	bibstyle=ieee,
	citestyle=numeric-comp,
	dashed=false,
	url = false,
	maxnames=8,
	maxcitenames=2,
	mincitenames=1,
	sorting=none,
	isbn = false,
	doi = false
	]{biblatex}
	\addbibresource{references.bib}

	%Set the page style
	\pagestyle{fancy}
	\fancyhead[L]{\ifodd\value{page} \slshape\nouppercase{\rightmark} \else \fi}
	\fancyhead[R]{\ifodd\value{page} \else \slshape\nouppercase{\leftmark} \fi}
	\chead{ }
	\lfoot{}
	\rfoot{}
	\cfoot{\small\thepage}


	%Give colors to links/refs etc
	\hypersetup{colorlinks,	linkcolor={blue!0!black}, 
							citecolor={blue!70!black}, 
							 urlcolor={blue!80!}} 
						 
	%% Set up numbering and spacing
	\numberwithin{equation}{chapter}		%Number the equations per section
	\numberwithin{figure}{chapter}			%Number the figures per section
	\numberwithin{table}{chapter}			%Number the tables per section
	\captionsetup[table]{skip=1pt}			%Skip 1 pt after a table
	\captionsetup[figure]{skip=3.5pt}		%Skip 4 pt after a figure
	\setcounter{secnumdepth}{3}				%Count up to the subsubsection 
	\setcounter{topnumber}{1}				%Number of floats at top of a page (default is 2)
% DEFINITIONS
	%% Titlepage definitions
	\newcommand{\deltitle}{Experimental Validation of Reference Spreading for Robotic Manipulation of Unmodeled Objects}        %Your project title
	\newcommand{\StudentName}{A.A.H.M. van den Brandt}               %Student name
	\newcommand{\StudentID}{1257110}                    %Your student number
	\newcommand{\DCcode}{202x.xxx}                      %Get your DC code from the D&C secretariat

	%% Operators
	\DeclareMathOperator\sign{sgn}                      %Sign function
	\DeclareMathOperator\diag{diag}                     %Diagonal operator
	\DeclareMathOperator\imag{Imag}                     %Imaginary part of complex variable
	\DeclareMathOperator\real{Real}                     %Real part of complex variable
	\DeclareMathOperator*{\argmin}{\arg\!\min}          %Argmin operator
	\newcommand{\norm}[1]{\left\lVert#1\right\rVert}    %Norm operator

	%% Variable definition
	\newcommand{\R}{\mathbb{R}}                         % Set of real numbers
	\newcommand{\C}{\mathbb{C}}                         % Set of complex numbers

\begin{document}

% TITLE PAGE
	\hypersetup{pageanchor=false}
	\begin{titlepage}
	\centering
	\includegraphics[scale=1.9]{Graphics/TUE-logo.pdf}\\[0mm]
	\begin{center}
		{\Large Department of Mechanical Engineering}\\[2mm]
		{\Large Dynamics and Control section}\\[20mm]
	\end{center}

	% Project-specific information, see "DEFINITIONS" above
	\begin{center}
		{\Huge \deltitle}\\[2mm]
		{\Large Master thesis}\\[2mm]
		{\large \StudentName \ (\StudentID)}\\[1mm]
		{\large DC \DCcode}\\[70mm]
	\end{center}
		
	% Coaching, supervision and committee
	\flushleft\large {\textbf{Coach:}} \\
	{ir. J.J. van Steen}   \\    
	% {title. Int. Surname (External Institution, City, Country)}  \\ % for not TU/e faculty, mention explicitly the institution


	\flushleft\large {\textbf{Supervisors:}} \\
	{dr. ir. A. Saccon}             \\
	{dr. ir. J. Kober (TU Delf)}             \\

	\flushleft\large {\textbf{Additional MSc Committee members:}} \\
	{title. A. Surname}  
	\\~\\
	\small
	This report was made in accordance with the TU/e Code of Scientific Conduct for the Master thesis            

	% Location and date of writing
	\vfill\center Eindhoven, DD Mmm YYYY \\ % put date of defence
		
	\end{titlepage}
	
\end{document}
	\hypersetup{pageanchor=true}

	\thispagestyle{empty} \ \newpage								%Empty page
	\pagenumbering{roman}											%Set page numbering to Roman numerals
% ABSTRACT
	\chapter*{Abstract}
	\addcontentsline{toc}{chapter}{Abstract}						%Add Abstract to Contents list
	\lipsum[3-6]

	\newpage
	\thispagestyle{empty} \ \newpage								%Empty page
% ACKNOWLEDGEMENTS
	% Acknowledgements are optional, comment or remove this section if not preferred
	\chapter*{Acknowledgments}
	\addcontentsline{toc}{chapter}{Acknowledgements}    			%Add Acknowledgments to Contents list
	\lipsum[3-6]

	\newpage
	\thispagestyle{empty} \ \newpage								%Empty page
% TABLE OF CONTENTS
	\addcontentsline{toc}{chapter}{Contents}    			        %Add Contents to Contents list
	\tableofcontents\newpage 		 								%Table of contents
	\thispagestyle{empty} \ \newpage								%Empty page

% NOMENCLATURE
	\chapter*{Nomenclature}
	\addcontentsline{toc}{chapter}{Nomenclature}					%Add Nomenclature to contents list
	\markboth{Nomenclature}{Nomenclature} 
	\thispagestyle{empty}

	\textbf{Groups, algebras, and sets}\\[3mm]
	\begin{tabularx}{\textwidth}{p{2.5cm}X}
		$\C$            & The set of complex numbers                        \\
		$\R$            & The set of real numbers                           \\
		$\R^+$          & The set of non-negative real numbers      
	\end{tabularx}\\

	\vspace{0.5cm}
	\noindent\textbf{Roman symbols}\\[3mm]
	\begin{tabularx}{\textwidth}{p{2.5cm}X}
		$t$             & Global time                                       \\
		$x$             & State variable                                    
	\end{tabularx}

	\vspace{0.5cm}
	\noindent\textbf{Greek symbols}\\[3mm]
	\begin{tabularx}{\textwidth}{p{2.5cm}X}
		$\Delta t$         & Time interval of a time-step                   \\
		$\delta$           & Dirac delta function                           \\
		$\delta_k$         & Kronecker delta function                       
	\end{tabularx}

	\vspace{0.5cm}
	\noindent\textbf{Subscripts and superscripts}\\[3mm]
	\begin{tabularx}{\textwidth}{p{2.5cm}X}
		$\dot{(\cdot)}$    & First time derivative                          \\
		$\ddot{(\cdot)}$   & Second time derivative                         \\
		$\hat{(\cdot)}$    & Estimated or approximate variable              \\
		$\bar{(\cdot)}$    & Mean or expected value                         
	\end{tabularx}\\


	\hypersetup{linkcolor={blue!70!black}}		            		%Change the link color
	\newpage
	\thispagestyle{empty} \ \newpage

% INTRODUCTION
	\chapter{Introduction} \label{ch:Intro}
	\pagenumbering{arabic}						            		%Set numbering to arabic
	In logistics, robots are not yet able match humans in versatility and speed when it comes to object manipulation. Consider a conveyor belt with packages to be picked up: while humans inherently have the ability to handle packages of various geometries and inertias, significant engineering effort on a case-to-case basis such as modeling the package or generating a reference is required for robots to perform the varying picking tasks -- occasionally to the point where the usage of robots becomes infeasible. Furthermore, while a human's bodily structure and sense of touch allow for careful yet swift grabbing of objects, robots for object manipulation are still lacking on this front. In an impact, i.e., a change in velocity due to collision between two objects, the contact force peaks. This peak is more pronounced for higher velocities. Therefore, to prevent disruption or damage to robots and packages, the control approach in a picking task often involves slowing down prior to making contact. Not slowing down would increase productivity of a robot, but requires adequate hard- and software.

% CHAPTER 2
	\chapter{Background}
	\section{Hardware for compliant robots}
	Historically, stiffness has been an important aspect for robots. Stiffness allows for an easy method to reject disturbances such as friction or hindering objects. To achieve a high stiffness with limited actuator force or torque, robots typically use drivetrains with high gear ratios. Once disturbances become large, high gear ratios become problematic. These drivetrains tend to have a low backdriveability; applying a force at the linkside of the drivetrain generally causes a lot of friction and little force transmitted to the motorside. This friction makes the drivetrain susceptible to breaking. Furthermore, tracking with high a stiffness results on large contact forces being exerted on heavy objects that block the path.

	It can be concluded that stiff robots with non-backdriveable gearboxes are not suitable for high-velocity impacts with objects whose location is uncertain or unknown. Instead, the robot must be compliant have a high backdriveability. One way to improve backdrivability is to use torque sensors combined with flexible elements between the drivetrain and the linkside. Even if the drivetrain blocks when a torque applied on the linkside, the flexible element still allows for some compliance. Furthermore the torque sensors is used to in closed loop so that the motor rotates until the desired torque is achieved. This method allows robots with non-backdrivable gearboxes, such as the KUKA LBR or Franka Emika Robot, to be backdrivable. Another method


% CONCLUSIONS
	\chapter{Conclusion and recommendations}
	\section{Conclusion}
	\lipsum[3]

	\section{Recommendations}
	\lipsum[3]

	\newpage
	\thispagestyle{empty} \ \newpage
% BIBLIOGRAPHY
	\addcontentsline{toc}{chapter}{References}
	\printbibliography[title=References]

	\newpage
	\thispagestyle{empty} \ \newpage

%APPENDIX A
	\appendix
	\chapter{Appendix A Title}
	Appendixes should appear after your conclusions and bibliography.
	\section{Section of the first appendix}
	\lipsum[3]


	\newpage
	\thispagestyle{empty} \ \newpage
%APPENDIX B 
	\chapter{Appendix B Title}
	\lipsum[2]

	\section{A section Title}
	\lipsum[3]

\end{document}